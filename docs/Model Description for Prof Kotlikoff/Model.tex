\hypertarget{model}{%
\subsection{Model}\label{model}}

\hypertarget{demography}{%
\subsubsection{Demography}\label{demography}}

In each year, there are \(S = 80\) generations alive. Each agent enters
the system at 20 years old (\(s = 1\)), works for \(S_{r} = 40\) years
then retire at end of the year. The population in year \(t\) can be
denoted as:

\[N_{t} = \sum_{s = 1}^{S}N_{s,t}\]

Without income groups, the number of people born in year \(t\) is
non-increasing:

\[N_{s + 1,t + s} = N_{s,t + s - 1}\left( 1 - F_{s,t + s - 1} \right)\]

where \(F_{s,t + s - 1} \in \left\lbrack 0,1 \right)\) is mortal
probabilities. The data are profiled by population forecast.

In each working year, agents have 1 time endowment and make their
leisure decision \(l_{s,t} \in \left\lbrack 0,1 \right\rbrack\). The
total labor supply in year \(t\) is:

\[L_{t} = \sum_{s = 1}^{S_{r}}{N_{s,t}\left( 1 - l_{s,t} \right)}\]

\hypertarget{firm}{%
\subsubsection{Firm}\label{firm}}

We use a Cobb-Douglas production function:

\[Y_{t} = A_{t}{K_{t}}^{\beta}{L_{t}}^{1 - \beta}\]

When equilibrium reached, the firm makes its optimal decision:

\[r_{t} = \frac{\partial Y_{t}}{\partial K_{t}} - \kappa\]

\[{\overline{w}}_{t} = \frac{\partial Y_{t}}{\partial L_{t}}\]

where \(\kappa\) is depreciation rate.

Wage level varies by age. A relative relationship
\(\varepsilon\left( s \right)\) and a scaling parameter \(o_{t}\) are
used to profile wages for the working people in each year.
\(\varepsilon\left( s \right)\) is profiled by survey data.

\[w_{s,t} = {\overline{w}}_{t}o_{t}\varepsilon\left( s \right)\]

\[{\overline{w}}_{t}L_{t} = \sum_{s = 1}^{S_{r}}{w_{s,t}N_{s,t}\left( 1 - l_{s,t} \right)}\]

\hypertarget{social-security}{%
\subsubsection{Social Security}\label{social-security}}

There are two social security plans: a pay-as-you-go pension and a mixed
social medical security system.

\hypertarget{pension}{%
\paragraph{Pension}\label{pension}}

It is convenient to have a simplified, total pay-as-you-go pension in
our discussion about medical system. In China, agents and firms
contribute to the pension together. In each year of working, the firm
pre-contributes \(z_{t}\eta_{t}\) from agent's wage to the pension and
contributes \(\zeta_{t}\) to the medical system, where \(z_{t}\) is
collection rate of pension:

\[w_{s,t} = \left( 1 + z_{t}\eta_{t} + \zeta_{t} \right){\widetilde{w}}_{s,t}\]

When agents receive their real wage level \({\widetilde{w}}_{s,t}\),
they contribute extra \(z_{t}\theta_{t}\) to the pension. The
\(\theta_{t}\) works on \({\widetilde{w}}_{s,t}\). We use
\(\pi_{t} = \frac{z_{t}\left( \theta_{t} + \eta_{t} \right)}{1 + z_{t}\eta_{t} + \zeta_{t}}\)
as the total contribution rate to pension, then the pay-as-you-go
pension system can be denoted as:

\[\sum_{s = 1}^{S_{r}}{\pi_{t}w_{s,t}N_{s,t}\left( 1 - l_{s,t} \right)} = \sum_{s = S_{r} + 1}^{S}{\Lambda_{s,t}N_{s,t}}\]

where \(\Lambda_{s,t} = \Lambda_{t}\) is the pension benefit in year
\(t\). We make it distribute even on ages.

\hypertarget{medical-system}{%
\paragraph{Medical System}\label{medical-system}}

\textbf{The social medical security system consists of individual
accounts for each generation and one pooling account}. Each agent opens
his/her own individual account when born and closes the account at
death. The individual accounts are accumulative. There is also a
pay-as-you-go pooling account whose gap is covered by the government
budget in each year.

\textbf{In the medical market, there are two types of medical consumer
goods}. We assume a well-supplied medical market where agents can
consume any amount of medical service as they like\footnote{We do not
  model a medical production department and do not clearly distinguish
  the medical consumer goods from general consumer goods.}. For a
specific agent in year \(t\) at age \(s\), he/she decides his/her
consumption \(c_{s,t}\) in a perfect-forward-looking utility
optimization (explained in the next section). Meantime, his/her total
medical consumption \(m_{s,t}\) is also determined by a ratio
\(q_{t} = \frac{m_{s,t}}{c_{s,t}}\). The \(q_{t}\) ratios are exogeneous
in steady states but time-variant on the transition path. It varies
according to a constant income elasticity of medical expenditure:
\(x = \frac{\partial M_{t}}{M_{t}}\frac{Y_{t}}{\partial Y_{t}} = 1.6\),
where \(Y_{t}\) is GDP and \(M_{t} = \sum_{s = 1}^{S}{m_{s,t}N_{s,t}}\)
is the social total medical expenditure in year \(t\). However, we set a
cap of \(q_{t}\) to avoid a unreasonable level. When \(m_{s,t}\)
defined, we use another exogenous ratio
\(p_{s,t} = \frac{\text{MA}_{s,t}}{\text{MB}_{s,t}}\) to distinguish the
outpatient expenditure \(\text{MA}_{s,t}\) from the inpatient
expenditure \(\text{MB}_{s,t}\).

\textbf{The two types of accounts separately cover the two types of
medical expenditures}. Based on the case of China, we have the pooling
account to cover \(1 - \text{cp}_{t}^{B}\) of agents' inpatient
expenditures, where \(cp_{t}^{B}\) is the copayment rate. The left
\(\text{cp}_{t}^{B}\) part is paid by agents' personal asset account
\(a_{s,t}\). The individual medical accounts, together with agents'
personal asset accounts, pay for the outpatient expenditures
\(\text{MA}_{s,t}\). An agent will primarily use the money in his/her
individual medical account to pay for \(\text{MA}_{s,t}\). If no money
left there, the agent will use his/her personal asset account to pay the
bill.

\textbf{Contributions to the two types of medical accounts are designed
to meet the reality of China}. In working years, the firm contributes
\(\zeta_{t}\) of \({\widetilde{w}}_{s,t}\) to the medical system. Then
each agent contributes extra \(\phi_{t}\) of \({\widetilde{w}}_{s,t}\)
to his/her individual medical account. However, there are
\(\mathbb{a}_{t}\) of firm's contribution transferred to individual
medical accounts of those working agents, \(\mathbb{b}_{t}\) of firm's
contribution transferred to individual medical accounts of those retired
agents. The left money is then contributed to the pooling medical
account. We use the following figure to intuitively show the
relationships in a specific year:

Figure Contributions \& Payments of Medical System

where \(\sigma\) is wage tax rate,
\(\pi_{t}^{M} = \frac{\phi_{t} + \zeta_{t}}{1 + z_{t}\eta_{t} + \zeta_{t}}\)
is the total contribution rate to medical system, \(LI_{t}\) is the gap
of pooling medical account, and \(\Phi_{s,t}\) is the balance of agents'
individual medical account. If we write the pooling medical account in
mathematics, we have:

\[\sum_{s = 1}^{S}{\frac{1 - \text{cp}_{t}^{B}}{1 + p_{t}}q_{s,t}c_{s,t}N_{s,t}} = \left( 1 - \mathbb{a}_{t} - \mathbb{b}_{t} \right)\frac{\zeta_{t}}{1 + z_{t}\eta_{t} + \zeta_{t}}\sum_{s = 1}^{S_{r}}{w_{s,t}\left( 1 - l_{s,t} \right)N_{s,t}} + \text{LI}_{t}\]

\textbf{The accumulation of individual medical accounts is similar to
personal asset accounts}. Agents supply their wealth (both \(a\) and
\(\Phi\)) to the capital market and get interest incomes. When agents
die before the life limit, their wealth will be redistributed evenly to
all agents at the same age. After mathematical simplification, we have:

\[\left\{ \begin{matrix}
\&\left( 2 - \frac{1}{1 - F_{s}} \right)\Phi_{s + 1} = \left( 1 + r_{s} \right)\Phi_{s} + \frac{\phi_{s} + \mathbb{a}_{s}\zeta_{s}}{1 + z_{s}\eta_{s} + \zeta_{s}}w_{s}\left( 1 - l_{s} \right) - \frac{q_{s}p_{s}}{1 + p_{s}}c_{s},s = 1,\ldots,S_{r} \\
\&\left( 2 - \frac{1}{1 - F_{s}} \right)\Phi_{s + 1} = \left( 1 + r_{s} \right)\Phi_{s} + \mathbb{P}_{s} - \frac{q_{s}p_{s}}{1 + p_{s}}c_{s},s = S_{r} + 1,\ldots,S \\
\end{matrix} \right.\ \]

where \(\frac{q_{s}p_{s}}{1 + p_{s}}c_{s} = \text{MA}_{s}\), and
\(\mathbb{P}_{s,t} = \mathbb{P}_{t}\) is the transfer amount from firm
contributions to those retired. This amount is determined by:

\[\mathbb{P}_{t}\sum_{s = S_{r} + 1}^{S}N_{s,t} = \sum_{s = 1}^{S_{r}}{\mathbb{b}_{t}\frac{\zeta_{t}}{1 + z_{t}\eta_{t} + \zeta_{t}}w_{s,t}\left( 1 - l_{s,t} \right)N_{s,t}}\]

\hypertarget{household}{%
\subsubsection{Household}\label{household}}

\textbf{Agents solve a perfect forward-looking utility optimization when
born}. The cross-section utility function is:

\[u\left( c,l \middle| q,\alpha,\gamma,\varrho \right) = \frac{1}{1 - \gamma^{- 1}}\left\lbrack \left( \left( 1 - q \right)c \right)^{1 - \varrho^{- 1}} + \alpha l^{1 - \varrho^{- 1}} \right\rbrack^{\frac{1 - \gamma^{- 1}}{1 - \varrho^{- 1}}}\]

where \(\alpha\) is the leisure preference than consumption, \(\gamma\)
is the inter-temporal substitution elasticity, and \(\varrho\) is the
consumption substitution elasticity of labour. Here we assume that only
non-medical consumptions bring utility to agents. The problem can be
written as a Bellman equation:

\[\nu_{s} = \max\left\lbrack u\left( c_{s},l_{s} \right) + \beta_{s}^{u}\nu_{s + 1} \right\rbrack\]

where \(\beta_{s}^{u} = \frac{1 - F_{s}}{1 - \delta}\) is discount
factor, and \(\delta\) is the utility discount rate.

\textbf{The payment of outpatient expenditures can be seen as a transfer
from individual medical account to personal asset account}. Agents
primarily use the money in their individual medical account \(\Phi\) to
cover the outpatient expenditures. Only if there is no money in the
account, agents will use the money in their personal account \(a\) to
pay the bill. However, for the purpose of generality, the process can be
described as: first, agents always transfer
\(MA = \frac{p}{\left( 1 + p \right)}\text{qc}\) from \(\Phi\) to \(a\),
then use \(a\) account to pay the outpatient bill. At last, if
\(\Phi < 0\), agents transfer money from \(a\) back to \(\Phi\) to make
sure \(\Phi = 0\). Considering bequests of accident death, we have
budget constraints:

\[\left\{ \begin{matrix}
\&\left( 2 - \frac{1}{1 - F_{s}} \right)a_{s + 1} = \left( 1 + r_{s} \right)a_{s} + \left( 1 - \sigma - \pi_{s} - \pi_{s}^{M} \right)w_{s}\left( 1 - l_{s} \right) - c_{s} + \frac{1}{1 + p_{s}}\left\lbrack p_{s} + \left( 1 - \text{cp}_{s}^{B} \right) \right\rbrack q_{s}c_{s},s = 1,..,S_{r} \\
\&\left( 2 - \frac{1}{1 - F_{s}} \right)a_{s + 1} = \left( 1 + r_{s} \right)a_{s} + \Lambda_{s} - c_{s} + \frac{1}{1 + p_{s}}\left\lbrack p_{s} + \left( 1 - \text{cp}_{s}^{B} \right) \right\rbrack q_{s}c_{s},s = S_{r} + 1,\ldots,S \\
\end{matrix} \right.\ \]

\textbf{Because we distinguish individual medical accounts, the boundary
conditions no longer work on personal asset but agents' wealth}:

\[\left\{ \begin{matrix}
\& a_{s = 1} = \Phi_{s = 1} = 0 \\
\& a_{\text{dead}} + \Phi_{\text{dead}} = 0 \\
\end{matrix} \right.\ \]

where \(a_{\text{dead}},\ \Phi_{\text{dead}}\) are the account balances
at the end of age year \(S\). If \(\Phi_{S} \neq 0\), we assume there is
a transfer in the last year to meet the assumption of no last bequest.

\hypertarget{government}{%
\subsubsection{Government}\label{government}}

The government keeps the following budget constraint:

\[\text{TR}_{t} + D_{t + 1} = G_{t} + \text{LI}_{t} + r_{t}D_{t}\]

where
\(TR_{t} = \sigma\sum_{s = 1}^{S_{r}}{N_{s,t}w_{s,t}\left( 1 - l_{s,t} \right)} + \mu\sum_{s = 1}^{S}{N_{s,t}c_{s,t}}\)
is the total tax revenues, \(\sigma\) is wage tax rate, \(\mu\) is
consumption tax rate, \(D_{t}\) is government outstanding debt. Further,
we have a soft constraint \(\frac{G_{t}}{Y_{t}} < \mathbb{k}_{t}\),
where \(\mathbb{k}_{t}\) is the cap of debt ratio, a given constant.

\hypertarget{equilibrium}{%
\subsubsection{Equilibrium}\label{equilibrium}}

For the good market, we have the following clearing condition:

\[Y_{t} = C_{t} + I_{t} + G_{t}\]

where \(C_{t} = \sum_{s = 1}^{S}{N_{s,t}c_{s,t}}\) is the total
consumption, and \(I_{t} = K_{t + 1} - \left( 1 - \kappa \right)K_{t}\)
is the investment.

For the capital market, we have the clearing condition:

\[K_{t} = \sum_{s = 1}^{S}{N_{s,t}\left( a_{s,t} + \Phi_{s,t} \right)} + D_{t}\]

\hypertarget{notations}{%
\subsection{Notations}\label{notations}}

\begin{longtable}[]{@{}ll@{}}
\toprule
\textbf{Marks} & \textbf{Names}\tabularnewline
\midrule
\endhead
\(o_{t}\) & Scaling coefficient for wage profiling\tabularnewline
\(\varepsilon\left( s \right)\) & Relative wage level\tabularnewline
\(\text{LI}_{t}\) & Gap of pooling medical account\tabularnewline
\(\mu,\sigma\) & Consumption, Wage tax rate\tabularnewline
\(\kappa\) & Depreciation rate\tabularnewline
\(\theta,\eta\) & Personal, Firm contribution to pension\tabularnewline
\(\phi,\zeta\) & Personal, Firm contribution to medical\tabularnewline
\(\Lambda\) & Pension benefit\tabularnewline
\(\pi,\pi^{M}\) & Total contribution of pension, medical\tabularnewline
\(F\) & Mortal probabilities\tabularnewline
\(q\) & Ratio of medical fee to total consumption\tabularnewline
\(p\) & Ratio of Outpatient fee to Inpatient fee\tabularnewline
\(a,\Phi\) & Personal asset, Individual medical account\tabularnewline
\(\text{cp}^{B}\) & Copayment rate of inpatient fee\tabularnewline
\(\gamma\) & Inter-temporal substitution elasticity\tabularnewline
\(\alpha\) & Preference of leisure than consumption\tabularnewline
\(\varrho\) & Consumption substitution elasticity of
labour\tabularnewline
\(\mathbb{a}\) & Transfer rate from firm contribution to individual
medical account (working phase)\tabularnewline
\(\mathbb{b}\) & Transfer rate from firm contribution to the individual
medical account of those retired\tabularnewline
\(\mathbb{k}\) & Cap of debt to GDP ratio\tabularnewline
\(\mathbb{P}\) & Transferred amount from firm contribution to the
individual medical account of the retired\tabularnewline
\bottomrule
\end{longtable}

Table Notations
